\chapter{Introduction}

Originally, server applications were running on dedicated physical servers. If the application needed more resources, the server would be upgraded with additional CPUs, memory, or storage which  required physical access and only allowed adding whole new disk, or whole new memory board. Virtualisation (running virtual machines on a single physical host) solved both limitations, as the virtual machines could be managed remotely and only small chunks of storage or memory could be added. Cloud computing is another step in the evolution, as it creates an abstract layer of resources over multiple servers and provides advanced techniquies such as scheduling and load balancing. All of this is described in more detail in section \ref{se:cloud-intro}.

OpenStack is a cloud platform with distributed architecture, and is briefly described in section \ref{se:openstack-intro}. Deploying OpenStack can be a complex task requiring deep knowledge in distributed system architecture, networking, storage, and virtualisation. Because of its complexity, it is very hard to create an installation script. However, there are several projects that try to simplify the process. They are described in chapter \ref{ch:existing}.

However, before I start describing the methods of installing OpenStack, the reader should understand automatic application deployments which is described in section \ref{ch:ansible} using technology called Ansible.

As said before, OpenStack is a complex system with distributed architecture. It is described in more detail in chapter \ref{ch:arch} which describes the architecture of openstack and its components.

The goal of this thesis is to design and implement an installation script which would enable to install basic OpenStack deployment and would also offer flexibility in configuring more complex scenarios by reusing the individual components of the script. The design of a reference OpenStack architecture and the script is described in chapter \ref{ch:implementation}.

The final chapter \ref{ch:testing} shows some details about the implementation process, describes the testing environment and will also present results of the testing.

\section{Relation with Red Hat}

This thesis has been written in cooperation with Red Hat\footnote{https://www.redhat.com}, a leading open-source technology company. Some parts of the installation script designed and created as part of this thesis will be reused by the Fedora Project\footnote{https://getfedora.org} in its own infrastructure.

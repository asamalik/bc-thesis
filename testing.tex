\chapter{Implementation and Testing}
\label{ch:testing}


\section{Third-party Modules in the Playbook}

The implementation requires two third-party modules for managing Keystone endpoints and services. Both modules have been created by Davide Guerri <davide.guerri@gmail.com>, licensed under Apache License, Version 2.0, and are included as part of the playbook.

\section{Testing of the Deployment}

I have prepared the following setup to for testing the environ
\subsection{Description of the Host Environment}

The reference testing environment consists of three virtual machines with the following specification:

\subsubsection*{Controller}

\begin{itemize}
  \item{CPUs: 2}
  \item{RAM: 2048 MB}
  \item{Disk: 30 GB}
\end{itemize}

\subsubsection*{Compute}

\begin{itemize}
  \item{CPUs: 4}
  \item{RAM: 6144 MB}
  \item{Disk: 30 GB}
\end{itemize}

\subsubsection*{Storage}

\begin{itemize}
  \item{CPUs: 1}
  \item{RAM: 1536 MB}
  \item{Disk1: 20 GB}
  \item{Disk2: 60 GB}
\end{itemize}

These virtual machines were running on a laptop with the following configuration:

\begin{itemize}
  \item{CPU: 2.8 GHz Intel i7 4980HQ}
  \item{RAM: 16 GB}
  \item{SSD: PCIe 3.0 x4 8.0 GT/s (25.6 Gbit/s)}
\end{itemize}

\subsection{Using Ansible with Vagrant}

Vagrant is a tool that manages virtual machines for development environment. It uses private key authentication and generates its own keys for the virtual machines.

First, Vagrant needs to be configured to use a single private key to authenticate to all three virtual machines. To achieve this, the following line needs to be put in the \texttt{Vagrantfile}:


\begin{lstlisting}
config.ssh.insert_key = false
\end{lstlisting}



Vagrant routes SSH ports of the guest virtual machines to the localhost and uses different port for each. A command \texttt{vagrant ssh-config} will show the ports of each virtual machine. These ports need to be used in the \texttt{hosts} file. An example might look like this:

\begin{lstlisting}
controller ansible_ssh_host=127.0.0.1 ansible_ssh_port=2222
compute1 ansible_ssh_host=127.0.0.1 ansible_ssh_port=2200
compute1 ansible_ssh_host=127.0.0.1 ansible_ssh_port=2201
\end{lstlisting}

The last step is to configure Ansible to use the correct user name and private key. This can be done by creating file called \texttt{ansible.cfg} with the following content:

\begin{lstlisting}
[defaults]
hostfile = hosts
remote_user = vagrant
private_key_file = ~/.vagrant.d/insecure_private_key
host_key_checking = False
\end{lstlisting}

\subsection{Running the Deployment}

To use the Ansible Playbook, the following command needs to be issued:

\begin{lstlisting}
ansible-playbook deploy-openstack.yml
\end{lstlisting}

The deployment of all three nodes takes approximately 7 minutes and requires internet connection.
